% For MISTA 2015, use the default option that has been supplied
\documentclass{svjour3}                     % onecolumn (standard format)
%\documentclass[smallextended]{svjour3}     % onecolumn (second format)
%\documentclass[twocolumn]{svjour3}         % twocolumn
%
\smartqed  % flush right qed marks, e.g. at end of proof
%
\usepackage{graphicx}
\usepackage{amsmath,amssymb}
%
% \usepackage{mathptmx}      % use Times fonts if available on your TeX system
%
% insert here the call for the packages your document requires
%\usepackage{latexsym}
% etc.
%
% please place your own definitions here and don't use \def but
% \newcommand{}{}
%
% Insert the name of "your journal" with
% This is preset for MISTA 2015: Do not change
\journalname{MISTA 2015}
%
\begin{document}

\title{Model for planning of distributed data production}

\author{Dzmitry Makatun         \and
		J\'er\^ome~Lauret		\and
		Hana~Rudov\'a			\and
		Michal~\v{S}umbera	
}

\institute{Dzmitry Makatun \at
              Faculty of Nuclear Physics and Physical Engineering, Czech Technical University in Prague \\
              \email{dzmitry.makatun@fjfi.cvut.cz}           %  \\
           \and
           J\'er\^ome~Lauret \at
              STAR, Brookhaven National Laboratory (BNL), USA \\
              \email{jlauret@bnl.gov}          %  \\              
           \and
           Hana~Rudov\'a \at
              Masaryk University, Brno, Czech Republic \\
              \email{hanka@fi.muni.cz}           %  \\              
           \and
           Michal~\v{S}umbera \at
              Nuclear Physics Institute (NPI), Academy of Sciences (ASCR),
 Czech Republic \\
              \email{sumbera@ujf.cas.cz}           %  \\
}

\maketitle
\section{Introduction}
\label{intro}
The STAR experiment at the Relativistic Heavy Ion Collider (RHIC) studies a
primordial form of matter that existed in the universe shortly after the Big
Bang. Collisions of heavy ions occur millions of times per second inside the
detector, producing tens of petabytes of raw data each year. All the raw data
has to be processed in order to reconstruct physical events which are
further analyzed by scientists. This process is called data production.  Like
any other modern experiment in High Energy and Nuclear Physics (HENP), STAR intends to rely on
distributed data processing, making use of several remote computational sites
(for some experiments this number can scale up to several hundreds).

When running data intensive applications on distributed computational
resources long I/O overheads may be observed as access to remotely stored data
is performed. Latency and bandwidth can become the major limiting factors for
the overall computation performance and can reduce the CPU time\,/\,wall time 
ratio due to excessive I/O wait. 
Widely used data management systems in HENP community
(Xrootd, DPM) are focused on providing heterogeneous access to distributed
storage and do not consider data pre-placement with respect to available CPUs,
job durations or network performance. At the same time job scheduling systems
(PBS, Condor) do not reason about transfer overheads when accessing data at
distributed storage. For this reason, an optimization of data transferring and
distribution across multiple sites is often done manually, using a custom
setup for each particular infrastructure \cite{Balewski_2012pa}. 

In previous collaboration between BNL and NPI/ASCR, we addressed the problem of
efficient data transferring in a Grid environment \cite{Zerola}.
% and cache management \cite{Makatun_cache}. 
Data transfers between $n$~computational sites and $m$~data locations were
considered but job scheduling was not covered
by that work. In \cite{ACAT_2014} we
proposed a constraint programming planner that schedules computational jobs
and data transfers in a distributed environment in order to optimize resource
utilization and reduce the overall completion time. Since such global
scheduling is computationally demanding it should be divided into several
stages in order to improve scheduler performance and scalability. A planing of
resource load can be completed in the first stage before scheduling file
transfers and jobs. In this work we address the problem of data production
planning, answering the question how the data should be transferred given the
network structure, bandwidth, storage and CPU slots available. This will allow
local schedulers to process jobs and have CPUs busy all the time while not
exceeding disk and network capacities.

Optimization of data intensive applications in Grid was studied
in~\cite{Globus_scheduler}. In this work an optimization was achieved by
replication of highly used files to more sites while the jobs were executed
where their input data is located. However, this is not the case for data
production, when each file has to be processed once. 
%
Explicit model distributing jobs over a Grid with respect to the network
bandwidth was proposed in~\cite{Trees}. The network structure of the Grid was
modeled as a tree and all the files were assumed to be of the same size and
processing time. In our study we do not limit the network topology to trees,
and assume fluctuations of job parameters. 

\section{Modeling}
\label{modeling}
Due to a data level of parallelism a typical workflow of HENP computation
consists of independent jobs using one CPU, one input and one output file. We
assume there is a local scheduler running at each site, which picks a new
input file to process from the local storage of that site each time when a CPU becomes
free. Input data must be transferred from the central storage
to each site in such a manner that at the every moment of time there is enough
input files at each site to keep all the available CPUs busy while not
exceeding the local storage and network throughput. Another task
is to transfer the output files back to central storage, cleaning each local
storage for the new input.

Let us consider a scheduling time interval $\Delta T$. We assume that at the
starting moment all the CPUs in the Grid are busy, and there is some amount of
input data already placed at each site. We need to transfer the next portion
of data to each site during time interval $\Delta T$ in order to avoid
draining of the local queue by the end of this interval. 

The computational Grid is represented by a directed weighted graph where
vertexes $c_{i} \in C$ are computational nodes and edges $l_{j} \in L$ are
network links. Weight of each link $b_{j}$ is the amount of data that can be
transferred over the link per unit of time (i.e. bandwidth). One of the nodes
$c_{0}$ is the central storage where all the input files for the further
processing are initially placed. All the output files has to be transferred
back to $c_{0}$ from the computational nodes. We will give two separate
problem formulations: for an input and output transfer planning. 

In order to formulate a network flow maximization problem \cite{Network_flows}
for input/output file transferring we have to define a capacitated $\{s,t\}$
network, which is a set of vertexes $V$ including a source $s$ and a sink $t$;
and a set of edges $e\in E$ with their capacities $cap(e)$. A solution that
assigns a nonnegative integer number $f(e)$ to each edge $e \in E$ can be
found in polynomial time with known algorithms.

In order to transform a given graph of a Grid into a capacitated $\{s,t\}$
network for an input transfer problem we add two dummy vertexes: a source $s$
and a sink $t$. Next we add  dummy edges $d_{i} \in D$ from each computational
node $i$ to the sink, and a dummy edge $q_{0}$ from the source $s$ to the
central storage $c_{0}$. These dummy edges allow us to introduce constraints
on the storage capacity of the nodes. The set of vertexes $V$ consists of
computational nodes $C$ and dummy vertexes: $V= C \cup \{s,t\}$. The final set
of edges consists of real network links $L$, dummy edges $D$ from
computational nodes to the sink and from the source to the central storage
$q_{0}$: $E= L \cup D \cup \{q_{0}\}$. Capacity of each edge defines the
maximal amount of data that can be transferred over an edge within time
interval $\Delta T$: 
%
\begin{equation} 
\label{edge_cap} 
cap(e) = \left\{
\begin{array}{l l} 
b_{j} \cdot \Delta T & \text{if }e = l_{j} \in L \\ w_{i} &
\text{if } e = d_{i} \in D\\ k_{0} & \text{if } e = q_{0} 
\end{array} \right.
\end{equation} 
%
where $w_{i}$ is the maximal amount of data that can be
transferred to the node $i$ without exceeding its storage capacity $Disk_{i}$
and $k_{0}$ is the total size of available input files at $c_{0}$. We denote
the solution for the input transfer problem as $f^{in}(e)$.

For transfer of output files we use a similar transformation, but swap the
source $s$ and the sink $t$, change the direction of dummy edges and redefine
capacities of dummy edges. In this case the capacity $\overline{k}_{0}$ of the
dummy edge $\overline{q}_{0}$ leading from the central storage $c_0$ to the
sink $s$ is equal to the amount of data which can be transferred to $c_0$
within time interval $\Delta T$ (it is limited by the available space at the
central storage). The capacity $\overline{w}_{i}$ of dummy edges
$\overline{d}_{i}$ leading from the source $t$ to computational nodes $c_{i}$
is equal to the maximum amount of output data which can be transferred from
the node $c_{i}$.
%
\begin{equation}
\label{edge_cap_out}
cap(e) = \left\{ 
  \begin{array}{l l}
    b_{j} \cdot \Delta T & \text{if }e = l_{j} \in L \\
    \overline{w}_{i} & \text{if } e = \overline{d}_{i} \in \overline{D}\\
    \overline{k}_{0} & \text{if } e = \overline{q}_{0}
  \end{array} \right.
\end{equation}
%
We denote the solution for the output transfer problem as $f^{out}(e)$.

Let us consider data production jobs which perform the same type of processing
on the same type of files. Duration $p_{j}$ of a job $j$  processing an input
file of size $InSize_{j}$ at a node $i$ is $p_{j} = \alpha_{i} \cdot
InSize_{j}$ where $\alpha_{i}$ is constant for each node $i$.  The ratio of
size of input $InSize_{j}$ and output $OutSize_{j}$ files of each job $j$ is
considered to be constant for the same type of data processing, i.e.,
$OutSize_{j} = \beta \cdot InSize_{j}$.  During the time interval $\Delta T$ a
node $i$ with $NCPU_{i}$  of CPUs  will process $\frac{1}{\alpha_{i}} \cdot
NCPU_{i} \cdot \Delta T$ of input data and will produce
$\frac{\beta}{\alpha_{i}} \cdot NCPU_{i} \cdot \Delta T$ of output data.
Using constraints on storage space, we can define the maximal amount of input
$w_{i}$ and output $\overline{w}_{i}$ data which can be transferred to/from a
node $i$:
%
\begin{eqnarray}
w_{i} &=&
Disk_{i} - I_{i}^{in} - I_{i}^{out} + \frac{1 - \beta}{\alpha_{i}} \cdot
NCPU_{i} \cdot \Delta T + Del_{i}^{out} \label{w}\\
\overline{w}_{i} &=& I_{i}^{out} + \frac{\beta}{\alpha_{i}} \cdot NCPU_{i} \cdot \Delta T - Min_{i}^{out} \label{sigma}
\end{eqnarray}  
%
where $Disk_{i}$ is available disk space at the node $i$, $I_{i}^{in}$ and
$I_{i}^{out}$ are the initial size of input and output data at a local storage
respectively, $Del_{i}^{out}$ is the amount of output data that will be
transferred out of the node and deleted from its storage during $\Delta T$;
$Min_{i}^{out}$ is the total size of output files which cannot be transferred
because the jobs which produce them are not finished (output files of running
jobs). 

In the Eqn.~\ref{w}, $Del_{i}^{out}$ is equal to the amount of data which will
be transferred from a node $c_{i}$, i.e., the solution to the output transfer
problem $f^{out}(\overline{d}_{i})$. In the Eqn.~\ref{sigma}, $\Delta T$ and
$Min_{i}^{out}$ are parameters of the scheduler. The other values used in
Eqns.~\ref{w}--\ref{sigma} can be obtained from monitoring data right before
each planning iteration.

\section{Solving Procedure}
\label{solve}

It can be proven that the maximum flow problems for input and output transfers
can be solved independently under assumptions: (a) all the real network links
in the considered Grid are full-duplex, i.e., a network throughput between two
nodes is the same in both directions (b) in a steady state the size of the
output transferred from each node is proportional to the size of the input
transferred to that node in each scheduling interval, i.e.,
$f^{out}(\overline{d}_{i})= \beta \cdot f^{in}(d_{i})$, where $\beta \leq 1$.

Since in real environment the assumption (b) will not strongly hold due to
resource performance fluctuations we propose the following approach to
solve the problem:
%
\begin{enumerate}
\item Calculate values for $\overline{w}_{i}$ using Eqn.~\ref{sigma}.
\item Solve the problem for output data flows to obtain $f^{out}(e)$.
\item Using Eqn.~\ref{w} and $Del_{i}^{out} = f^{out}(\overline{d}_{i})$ calculate $w_{i}$.
\item For real links $l \in L$ reduce the capacity by the amount which is used by output transfers: $cap(l_{j}) = b_{j} \cdot \Delta T - f^{out}(l_{j})$.
\item Solve the problem for input transfers with $w_{i}$ and $cap(l_{j})$ defined in previous steps. Find input data flows $f^{in}(e)$.
\end{enumerate}
%
To conclude, this procedure is expected to compute feasible data transfers 
such that CPUs in Grid are busy with computational jobs while not exceeding 
local disk capacities.

\section{Conclusion}
\label{Conclusion}

In this paper we proposed a model of distributed data production, where all
the files from a single source has to be processed once and transferred back.
This model allows planning of WAN, storage and CPU loads using the network
flow maximization approach. The proposed model will be used in a distributed
data production planner which is being developed. The planner will enable
automated and scalable planning and optimization of distributed computations
which are highly required in data intensive computational fields such as High Energy and Nuclear Physics.

\begin{acknowledgements}
This work has been supported by the Czech Science Foundation
(13-20841S, P202$/$12$/$0306),  the MEYS grant CZ.1.07/2.3.00/20.0207 of the European Social Fund (ESF) in the Czech Republic: “Education for Competitiveness Operational Programme” (ECOP) and the Office of Nuclear Physics within the U.S. Department of Energy.  
\end{acknowledgements}

\bibliography{bibliography}{}
\bibliographystyle{spmpsci}

\end{document}

\end{document}
