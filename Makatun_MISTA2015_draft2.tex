%%%%%%%%%%%%%%%%%%%%%%% file template.tex %%%%%%%%%%%%%%%%%%%%%%%%%
%
% This is a general template file for the LaTeX package SVJour3
% for Springer journals.          Springer Heidelberg 2006/03/15
%
% Copy it to a new file with a new name and use it as the basis
% for your article. Delete % signs as needed.
%
% This template includes a few options for different layouts and
% content for various journals. Please consult a previous issue of
% your journal as needed.
%
%
% 5th Mar 2012
% NOTES for MISTA 2015
% This template has been supplied for use for the MISTA 2015 conference
% In essence it is the same as the one supplied by SV, but with additional comments
% The reason this template is being used is to make it as easy as possible
% to be able to submit to the post conference special issue of the Journal
% of Scheduling
%%%%%%%%%%%%%%%%%%%%%%%%%%%%%%%%%%%%%%%%%%%%%%%%%%%%%%%%%%%%%%%%%%%
%
% First comes an example EPS file -- just ignore it and
% proceed on the \documentclass line
% your LaTeX will extract the file if required
\begin{filecontents*}{example.eps}
%!PS-Adobe-3.0 EPSF-3.0
%%BoundingBox: 19 19 221 221
%%CreationDate: Mon Sep 29 1997
%%Creator: programmed by hand (JK)
%%EndComments
gsave
newpath
  20 20 moveto
  20 220 lineto
  220 220 lineto
  220 20 lineto
closepath
2 setlinewidth
gsave
  .4 setgray fill
grestore
stroke
grestore
\end{filecontents*}
%
% For MISTA 2015, use the default option that has been supplied
\documentclass{svjour3}                     % onecolumn (standard format)
%\documentclass[smallextended]{svjour3}     % onecolumn (second format)
%\documentclass[twocolumn]{svjour3}         % twocolumn
%
\smartqed  % flush right qed marks, e.g. at end of proof
%
\usepackage{graphicx}
\usepackage{amsmath,amssymb}
%
% \usepackage{mathptmx}      % use Times fonts if available on your TeX system
%
% insert here the call for the packages your document requires
%\usepackage{latexsym}
% etc.
%
% please place your own definitions here and don't use \def but
% \newcommand{}{}
%
% Insert the name of "your journal" with
% This is preset for MISTA 2015: Do not change
\journalname{MISTA 2015}
%
\begin{document}

\title{Model of distributed data-production in High Energy Physics.}

%\subtitle{Model of distributed data-production in terms of network flow maximization problem.}

\author{Dzmitry Makatun         \and
		J\'er\^ome~Lauret		\and
		Hana~Rudov\'a			\and
		Michal~\v{S}umbera	
}

\institute{Dzmitry Makatun \at
              Faculty of Nuclear Physics and Physical Engineering, Czech Technical University in Prague \\
              \email{dzmitry.makatun@fjfi.cvut.cz}           %  \\
           \and
           J\'er\^ome~Lauret \at
              STAR, Brookhaven National Laboratory, USA \\
              \email{jlauret@bnl.gov}          %  \\              
           \and
           Hana~Rudov\'a \at
              Masaryk University at Brno, Czech Republic \\
              \email{hanka@ics.muni.cz}           %  \\              
           \and
           Michal~\v{S}umbera \at
              Nuclear Physics Institute, Academy of Sciences,
 Czech Republic \\
              \email{sumbera@ujf.cas.cz}           %  \\
}

\maketitle
\section{Introduction}
\label{intro}
The STAR experiment at the Relativistic Heavy Ion Collider (RHIC) studies a primordial form of matter that existed in the universe shortly after the Big Bang. Collisions of heavy ions occur millions of times per second inside the detector, producing tens of petabytes of raw data each year. All the raw data has to be processed ones in order to reconstruct physical events which are further analyzed by scientists. This process is called data production.
Like any other modern experiment in High Energy Physics, STAR relies on distributed data processing, making use of several remote computational sites (for some experiments this number can scale up to several hundreds).

When running data-intensive applications on distributed computational resources long I/O overheads may be observed as access to remotely stored data is performed. Latency and bandwidth can become the major limiting factors for the overall computation performance and can reduce the CPU/WallTime ratio to excessive IO wait. Widely used in HEP community data management systems (Xrootd, DPM) are focused on providing heterogeneous access to distributed storage and do not consider data pre-placement with respect to available CPUs, job durations or network performance. At the same time job scheduling systems (PBS, CONDOR) do not reason about transfer overheads when accessing data at distributed storage. For this reason, an optimization of data transferring and distribution across multiple sites is often done manually, using a custom setup for each particular infrastructure \cite{Balewski:2012pa}. 

In previous collaborative work of BNL and NPI/ASCR, we addressed the problem of efficient data transferring in a Grid environment \cite{Zerola} and cache management \cite{Makatun:cache}. The transfer considered an optimization of moving data at N sites considering the data may be located at M locations. However, the problem of data transferring was considered disconnected from job scheduling. In \cite{ACAT 2014} we proposed a constraint programming based planner that schedules computational jobs, data placement (transfers) in a distributed environment in order to optimize resource utilization and reduce the overall processing completion time. Since such global scheduling of data processing over Grid is very computational demanding it should be divided into several stages in order to improve scheduler performance and scalability. Before scheduling particular file transfers and jobs a preemptive planing of resource load would be required. In this work we address the problem of data production planning, answering the question how the data should be transferred given the network structure, bandwidth, storage and CPU slots available. This planning model will be combined with previously developed scheduling approach.

Optimization of data-intense applications in Grid was studied by Ranganathan and Foster \cite{Globus: scheduler}. In their work an optimization was achieved by replication of highly used files to more sites while the jobs were executed where their input data is located. However, this is not the case for data production, where each file has to be processed ones. 

Beaumont et.al. in \cite{Trees} proposed an explicit model of how the jobs should be distributed over a Grid depending on the network bandwidth. But the network structure of Grid was modeled as a tree and all the files were assumed to be of the same size and processing time. In our study we do not limit the network topology to trees, and assume fluctuations of job parameters. 

\section{Modeling}
\label{modeling}
Due to a data level of parallelism a typical workflow of HEP computation consists of independent jobs using 1 CPU, one input and one output file. We assume there is a local scheduler running at each site, which picks a new input file to process from the local storage each time when a CPU slot becomes free. The part of the task is to transfer input data from the central storage to each site in such a manner that at the every moment of time there is enough input files at each site to keep all the available CPUs busy while not exceeding the local storage and network throughput. Another part of the task is to transfer the output files back to central storage, cleaning the local storage for the new input.

Let us consider a scheduling time interval $\Delta T$. We assume that at the starting moment all the CPUs in the Grid are busy, and there is some amount of input data already placed at each site. We need to transfer the next portion of data to each site during time interval $\Delta T$ in order to avoid draining of the local queue by the end of this interval. 

The computational Grid is represented with a directed weighted graph were vertexes $c_{i} \in C$ are computational nodes and edges $l_{j} \in L$ are network links. Weight of each link $b_{j}$ is the amount of data that can be transferred over the link per unit of time (i.e. bandwidth). One of the nodes $c_{0}$ is the central storage where all the input files for the further processing are initially placed. All the output files has to be transferred back to $c_{0}$ from the computational nodes. We will give two separate problem formulations: for an input and output transfer planning. 

In order to formulate a network flow maximization problem for input/output file transferring we have to define a capacitated $\{s,t\}$ network, which is a set of vertexes $V$ including a source $s$ and a sink $t$; and a set of edges $E$ with their capacities $cap(e)$. A feasible flow $f$ in such network is a function that assigns a nonnegative integer number $f(e)$ to each edge $e \in E$ such that:
\begin{equation}
\label{netflowprob} 
 \left\{ \begin{array}{l l}
 \forall e \in E: f(e) \leq cap(e) \\
 \forall v \in V-\{s,t\}: \sum\limits_{e \in In(v)} f(e) = \sum\limits_{e \in Out(v)} f(e)
  \end{array} \right.
\end{equation}  
The value of flow $f$, denoted $val(f)$, is the net flow leaving the source $s$, that is
\begin{equation}
\label{flowval}
val(f) = \sum\limits_{e \in Out(s)} f(e)
\end{equation} 
Known algorithms can solve this type of problems in polynomial time. 

In order to transform a given graph of a Grid into a capacitated $\{s,t\}$ network for an input transfer problem we add 2 dummy vertexes: a source $s$ and a sink $t$. Then we add  dummy edges $d_{i} \in D$ from each computational node $i$ to the sink, and a dummy edge $q_{0}$ from the source $s$ to the central storage $c_{0}$. These dummy edges allow us to introduce constraints on the storage capacity of the nodes. The set of vertexes $V$ consists of computational nodes $C$ and dummy vertexes: $V= C \cup \{s,t\}$. The final set of edges consists of real network links $L$, dummy edges $D$ from computational nodes to the sink and from the source to the central storage $q_{0}$: $E= L \cup D \cup \{q_{0}\}$. Capacity of each edge defines the maximal amount of data that can be transferred over an edge within time interval $\Delta T$: 
\begin{equation}
\label{edge_cap}
cap(e) = \left\{ 
  \begin{array}{l l}
    b_{j} \cdot \Delta T & \text{if }e = l_{j} \in L \\
    w_{i} & \text{if } e = d_{i} \in D\\
    k_{0} & \text{if } e = q_{0}
  \end{array} \right.
\end{equation}
where $w_{i}$ is the maximal amount of data that we can transfer to node $i$ without exceeding its storage capacity $Disk_{i}$, $k_{0}$ is the total size of available input files at $c_{0}$. We denote the solution for the input transfer problem as $f^{in}(e)$.

For transfer of output files we use a similar transformation, but swap the source $s$ and the sink $t$, change the direction of dummy edges and redefine capacities of dummy edges. In this case the capacity $\overline{k}_{0}$ of the dummy edge $\overline{q}_{0}$ leading from the central storage $c0$ to the sink $s$ is equal to the amount of data which can be transferred to $c0$ within time interval $\Delta T$ (it is limited by the available space at the central storage). The capacity $\overline{w}_{i}$ of dummy edges $\overline{d}_{i}$ leading from the source $t$ to computational nodes $c_{i}$ is equal to the maximum amount of output data which can be transferred from the node $c_{i}$.
\begin{equation}
\label{edge_cap_out}
cap(e) = \left\{ 
  \begin{array}{l l}
    b_{j} \cdot \Delta T & \text{if }e = l_{j} \in L \\
    \overline{w}_{i} & \text{if } e = \overline{d}_{i} \in \overline{D}\\
    \overline{k}_{0} & \text{if } e = \overline{q}_{0}
  \end{array} \right.
\end{equation}
We denote the solution for the output transfer problem as $f^{out}(e)$.

Let us consider data production jobs which perform the same type of processing on the same type of files. Duration $p_{j}$ of a job $j$  processing an input file of size $InSize_{j}$ at a node $i$ is
\begin{equation}
\label{alpha}
p_{j} = \alpha_{i} \cdot InSize_{j} 
\end{equation}
The ratio of size of input $InSize_{j}$ and output $OutSize_{j}$ files of each job $j$ is considered to be constant for the same type of data processing:
\begin{equation}
\label{beta}
OutSize_{j} = \beta \cdot InSize_{j} 
\end{equation}
During the time interval $\Delta T$ a node $i$ with $NCPU_{i}$  of CPUs  will process $\frac{1}{\alpha_{i}} \cdot NCPU_{i} \cdot \Delta T$ of input data and will produce $\frac{\beta}{\alpha_{i}} \cdot NCPU_{i} \cdot \Delta T$ of output data.
Using constraints on storage space, we can define the maximal amount of input $w_{i}$ and output $\overline{w}_{i}$ data which can be transferred to/from a node $i$:
\begin{equation}
w_{i} =
Disk_{i} - I_{i}^{in} - I_{i}^{out} + \frac{1 - \beta}{\alpha_{i}} \cdot NCPU_{i} \cdot \Delta T + Del_{i}^{out}
\label{w}
\end{equation}

\begin{equation}
\label{sigma}
\overline{w}_{i} = I_{i}^{out} + \frac{\beta}{\alpha_{i}} \cdot NCPU_{i} \cdot \Delta T - Min_{i}^{out}
\end{equation}  
where $Disk_{i}$ is available disk space at node $i$, 
$I_{i}^{in}$ and $I_{i}^{out}$ are the initial size of input and output data at a local storage respectively,
$Del_{i}^{out}$ is the amount of output data that will be transferred out of the node and deleted from its storage during $\Delta T$ ; 
$Min_{i}^{out}$ is the total size of output files which can not be transferred because the jobs which produce them are not finished (output files of running jobs). 

In the Equation \ref{w} $Del_{i}^{out}$ is equal to the amount of data which will be transferred from a node $c_{i}$, i.e. the solution to the output transfer problem $f^{out}(\overline{d}_{i})$. In the Equation \ref{sigma} $\Delta T$ and  $Min_{i}^{out}$ are parameters of the scheduler. The other values used in Equations \ref{w} - \ref{sigma} can be obtained from monitoring data right before each planning iteration.

\section{Solving Procedure}
\label{solve}
%\subsection{Solving procedure}
It can be proven that the maximum flow problems for input and output transfers can be solved independently under assumptions: (a) all the real network links in considered Grid are full-duplex, i.e. network throughput between two nodes is the same in both directions (b) in a steady state the size of the output transferred from each node is proportional to the size of the input transferred to that node in each scheduling interval, i.e. $f^{out}(\overline{d}_{i})= \beta \cdot f^{in}(d_{i})$, where $\beta \leq 1$.

Since in real environment the assumption (b) will not strongly hold due to resource performance fluctuations we propose the following approach for solving the problem:

\begin{enumerate}
\item Calculate values for $\overline{w}_{i}$ using Equation \ref{sigma}.
\item Solve the problem for output data flows to obtain $f^{out}(e)$.
\item Using Equation \ref{w} and $Del_{i}^{out} = f^{out}(\overline{d}_{i})$ calculate $w_{i}$.
\item For real links $l \in L$ reduce the capacity by the amount which is used by output transfers: $cap(l_{j}) = b_{j} \cdot \Delta T - f^{out}(l_{j})$.
\item Solve the problem for input transfers with $w_{i}$ and $cap(l_{j})$ defined in previous steps. Find input data flows $f^{in}(e)$.
\end{enumerate}

\section{Conclusion}
\label{Conclusion}
In this paper we proposed a model of distributed data-production, where all the files from a single source has to be processed ones and transferred back. This model allows planning of WAN, storage and CPU loads using the network flow maximization approach. The proposed model will be used in a global planner for Grid which is being developed. The planner will enable automated and scalable planning and optimization of distributed computations which is in high demand in data intensive computational fields such as the ones in the HEP communities.

\begin{acknowledgements}
This work has been supported by the Czech Science Foundation
(GA-CR)  and the Office of Nuclear Physics within the U.S.
Department of Energy.
\end{acknowledgements}

% BibTeX users please use one of
%\bibliographystyle{spbasic}      % basic style, author-year citations
%\bibliographystyle{spmpsci}      % mathematics and physical sciences
%\bibliographystyle{spphys}       % APS-like style for physics
%\bibliography{}   % name your BibTeX data base

% Non-BibTeX users please use
\begin{thebibliography}{}
%
% and use \bibitem to create references. Consult the Instructions
% for authors for reference list style.
%
%\bibitem{RefJ}
% Format for Journal Reference
%Author, Article title, Journal, Volume, page numbers (year)
% Format for books
%\bibitem{RefB}
%Author, Book title, page numbers. Publisher, place (year)
% etc


\bibitem{ACAT 2014}
Makatun D, Lauret J, Rudov\'a H and \v{S}umbera M
Planning for distributed workflows: constraint-based coscheduling of computational jobs and data placement in distributed environments \textit{J. Phys.: Conf. Ser.} ACAT conf. 2014

\bibitem{Makatun:cache}
Makatun D, Lauret J and \v{S}umbera M 2014 Study of cache performance in distributed environment for data processing,  \textit{J. Phys.: Conf. Series} \textbf{523} 012016

\bibitem{Zerola}
 Zerola M, Lauret J, Bart\'ak R and \v{S}umbera M 2012
 One click dataset transfer: toward efficient coupling of distributed storage resources and CPUs  \textit{J. Phys.: Conf. Ser.} \textbf{368} 
 
\bibitem{Balewski:2012pa}
  Balewski J, Lauret J, Olson D, Sakrejda I, Arkhipkin D, Bresnahan J, Keahey K and Porter J {\it et al.}
  Offloading peak processing to virtual farm by STAR experiment at RHIC 2012
  \textit{J. Phys. Conf. Ser.}  {\bf 368} (2012) 012011. 
 
\bibitem{Trees}
Beaumont O, Carter L, Ferrante J, Legrand A and Robert Y 2002
Bandwidth-Centric Allocation of Independent Tasks on Heterogeneous Platforms
\textit{International Parallel and Distributed Processing Symposium (IPDPS’2002),  IEEE Computer} Society Press

\bibitem{Globus: scheduler}
Ranganathan K and Foster I 2002
Decoupling computation and data scheduling in distributed data-intensive applications
\textit{11th IEEE International Symposium on High Performance Distributed Computing}pp 352 - 358

\end{thebibliography}

\end{document}
% end of file sample.tex

